\documentclass[10pt, openany]{book}
%
%  Packages to use
%
\usepackage{fancyhdr}
\usepackage{fancyvrb}
\usepackage{fancybox}
%
\usepackage{lastpage}
\usepackage{imakeidx}
%
\usepackage{amsmath}
\usepackage{amsfonts}
%
\usepackage{geometry}
\geometry{letterpaper}
%
\usepackage{url}
\usepackage{gensymb}
\usepackage{multicol}
\usepackage{xcolor}
%
\usepackage[pdf]{pstricks}
\usepackage{graphicx}
\DeclareGraphicsExtensions{.pdf}
\DeclareGraphicsRule{.pdf}{pdf}{.pdf}{}
%
% Rules to allow import of graphics files in EPS format
%
\usepackage{graphicx}
\DeclareGraphicsExtensions{.eps}
\DeclareGraphicsRule{.eps}{eps}{.eps}{}
%
%  Include the listings package
%
\usepackage{listings}
%
% Macro definitions
%
\newcommand{\operation}[1]{\textbf{\texttt{#1}}}
\newcommand{\package}[1]{\texttt{#1}}
\newcommand{\function}[1]{\texttt{#1}}
\newcommand{\constant}[1]{\emph{\texttt{#1}}}
\newcommand{\keyword}[1]{\texttt{#1}}
\newcommand{\datatype}[1]{\texttt{#1}}
\newcommand{\uvec}[1]{\textnormal{\bfseries{#1}}}
\newcommand{\comment}[1]{{\color{red}{#1}}}
%
\newcommand{\docname}{Users's Manual for \\ Brent's Root Ada Package}
%
% Front Matter
%
\title{\docname}
\author{Brent Seidel \\ Phoenix, AZ}
\date{ \today }
%========================================================
%%% BEGIN DOCUMENT
\begin{document}
%
%  Header's and Footers
%
\fancypagestyle{plain}{
  \fancyhead[L]{}%
  \fancyhead[R]{}%
  \fancyfoot[C]{Page \thepage\ of \pageref{LastPage}}%
  \fancyfoot[L]{Ada Programming}
  \renewcommand{\headrulewidth}{0pt}%
  \renewcommand{\footrulewidth}{0.4pt}%
}
\fancypagestyle{myfancy}{
  \fancyhead[L]{\docname}%
  \fancyhead[R]{\leftmark}
  \fancyfoot[C]{Page \thepage\ of \pageref{LastPage}}%
  \fancyfoot[L]{Ada Programming}
  \renewcommand{\headrulewidth}{0.4pt}%
  \renewcommand{\footrulewidth}{0.4pt}%
}
\pagestyle{myfancy}
%
% Produce the front matter
%
\frontmatter
\maketitle
\begin{center}
This document is \copyright 2024 Brent Seidel.  All rights reserved.

\paragraph{}Note that this is a draft version and not the final version for publication.
\end{center}
\tableofcontents

\mainmatter
%========================================================
\chapter{Introduction}
This project is the root project for most of my other Ada projects.  It provides a namespace, \package{BBS}, that the other projects live under.  It also provides some common types, units, and conversions that the other projects can use.

\section{License}
This project is licensed using the ``Unlicense'' license.  It is released into the public domain.  You can take it and do with it whatever you like.  Though, if you are going to use it as your package, I'd ask that you change the package name to avoid name conflicts.  

THE SOFTWARE IS PROVIDED "AS IS", WITHOUT WARRANTY OF ANY KIND, EXPRESS OR IMPLIED, INCLUDING BUT NOT LIMITED TO THE WARRANTIES OF MERCHANTABILITY, FITNESS FOR A PARTICULAR PURPOSE AND NONINFRINGEMENT. IN NO EVENT SHALL THE AUTHORS BE LIABLE FOR ANY CLAIM, DAMAGES OR OTHER LIABILITY, WHETHER IN AN ACTION OF CONTRACT, TORT OR OTHERWISE, ARISING FROM, OUT OF OR IN CONNECTION WITH THE SOFTWARE OR THE USE OR OTHER DEALINGS IN THE SOFTWARE.
%========================================================
\chapter{How to Obtain}

This project is currently available on GitHub at \url{https://github.com/BrentSeidel/BBS-Ada}.

\section{Dependencies}
\subsection{Ada Libraries}
The following Ada libraries are used:
\begin{itemize}
  \item \package{Ada.Unchecked\_Conversion}
\end{itemize}
\subsection{Other Libraries}
There are no other dependencies.

%========================================================
\chapter{Usage Instructions}
This is a library of routines intended to be used by some program.  To use these in your program, edit your *\keyword{.gpr} file to include a line to \keyword{with} the path to \keyword{bbs.gpr}.  Then in your Ada code \keyword{with} in the package(s) you need and use the routines.

%========================================================
\chapter{API Description}
%--------------------------------------------------------------------------------------------------
\section{\package{BBS}}
This package defines the following types:
\begin{itemize}
  \item \datatype{bit} as \datatype{range 0 .. 1 with Size => 1}
  \item \datatype{int8} as \datatype{range -128 .. 127 with size => 8}
  \item \datatype{uint8} as \datatype{mod 2**8  with size => 8}
  \item \datatype{int16} as \datatype{-(2**15) .. 2**15 - 1 with size => 16}
  \item \datatype{uint16} as \datatype{mod 2**16 with Size => 16}
  \item \datatype{int32} as \datatype{-(2**31) .. 2**31 - 1 with Size => 32}
  \item \datatype{uint32} as \datatype{mod 2**32  with Size => 32}
  \item \datatype{int64} as \datatype{range -(2**63) .. 2**63 - 1  with Size => 64}
  \item \datatype{uint64} as \datatype{mod 2**64 with Size => 64}
\end{itemize}

And the following conversion functions between signed and unsigned:
\begin{lstlisting}
function uint8_to_int8 is
     new Ada.Unchecked_Conversion(source => uint8, target => int8);
function int8_to_uint8 is
     new Ada.Unchecked_Conversion(source => int8, target => uint8);
function uint16_to_int16 is
     new Ada.Unchecked_Conversion(source => uint16, target => int16);
function int16_to_uint16 is
     new Ada.Unchecked_Conversion(source => int16, target => uint16);
function uint32_to_int is
      new Ada.Unchecked_Conversion(source => uint32, target => Integer);
function int_to_uint32 is
      new Ada.Unchecked_Conversion(source => Integer, target => uint32);
function uint64_to_int64 is
     new Ada.Unchecked_Conversion(source => uint64, target => int64);
function int64_to_uint64 is
     new Ada.Unchecked_Conversion(source => int64, target => uint64);
\end{lstlisting}

%--------------------------------------------------------------------------------------------------
\section{\package{BBS.Units}}
This package defines a number of physical unites along with some conversions and operations.  The units fall into the following categories:

\begin{center}
\begin{tabular}{|l|l|l|}
\hline
Type & Prefix & Base Units\\
\hline
Length &  len & meters\\
Area & area & meters$^2$\\
Volume & vol & liters\\
Mass &  mass & kilograms\\
Force & force & Newtons\\
Temperature & temp & Celsius\\
Pressure & press & Pascal\\
Velocity & vel & m/s\\
Acceleration & acce & m/(s$^2$)\\
Angular & ang & radians\\
Rotation rate & rot & radians/second\\
Magnetic & mag & Gauss\\
Electromotive force & emf & Volt\\
Electrical current & curr & Amper\\
Electrical resistance & res & Ohms\\
Frequency & freq & Hertz\\
Time & time & Seconds.\\
\hline
\end{tabular}
\end{center}

The following data types are defined.  Note that these are all defined as \datatype{Float}.  Change as needed for your application.  More conversions and operations can be defined as needed.

When you define a datatype (for example \datatype{len\_m}), as a new \datatype{Float}, you get all the operations that apply to \datatype{Float}.  For addition and subtraction, that's fine since the result is also a length.  How a length times a length is an area, so multiplication needs to be redefined.  This package only scratches the surface of the possible operations.  Add more as needed for your application.

\subsection{Length Types}
The following length types are defined:
\begin{lstlisting}
type len_m is new Float;   -- length in meters
type len_ft is new Float;  -- length in feet
type len_A is new Float;   -- length in Angstroms
\end{lstlisting}

The following conversions are defined for the length types:
\begin{lstlisting}
function to_feet(dist : len_m) return len_ft;
function to_Angstroms(dist : len_m) return len_A;
function to_meters(dist : len_ft) return len_m;
function to_meters(dist : len_A) return len_m;
\end{lstlisting}

The following operations return a length type:
\begin{lstlisting}
function "*"(Left : vel_m_s; Right : Duration) return len_m;
function "*"(Left : Duration; Right : vel_m_s) return len_m;
\end{lstlisting}

\subsection{Area Types}
The following area type is defined:
\begin{lstlisting}
type area_m2 is new Float;  -- area in square meters
\end{lstlisting}

Since only one area type is defined, there are no conversions.  If more area types get added, appropriate conversions should also be added.

The following operation returns an area type:
\begin{lstlisting}
function "*"(Left, Right : len_m) return area_m2;
\end{lstlisting}

\subsection{Volume Types}
Volume types.  Prefix is ``vol''.  Base unit is liters.
\begin{lstlisting}
type vol_l is new Float;   -- volume in liters
type vol_m3 is new Float;  -- volume in cubic meters
\end{lstlisting}

The following conversions are defined for volume types:
\begin{lstlisting}
function to_liters(vol : vol_m3) return vol_l;
function to_meters3(vol : vol_l) return vol_m3;
\end{lstlisting}

The following operations return a volume type:
\begin{lstlisting}
function "*"(Left : len_m; Right : area_m2) return vol_m3;
function "*"(Left : area_m2; Right : len_m) return vol_m3;
\end{lstlisting}

\subsection{Mass Types}
Mass types.  Prefix is ``mass''.  Base unit is kilograms.
\begin{lstlisting}
type mass_kg is new Float;  -- mass in kilograms
type mass_lb is new Float;  -- mass in pounds
\end{lstlisting}

The following conversions are defined for mass types:
\begin{lstlisting}
function to_pounds(mass : mass_kg) return mass_lb;
function to_kilograms(mass : mass_lb) return mass_kg;
\end{lstlisting}

The following operations return a mass type:
\begin{lstlisting}
function "/"(Left : force_n; Right : accel_m_s2) return mass_kg
     with Global => null, pre => (Right /= 0.0);
\end{lstlisting}

\subsection{Force Types}
The following force type is defined:
\begin{lstlisting}
type force_n is new Float;  -- force in newtons
\end{lstlisting}

Since only one force type is defined, there are no conversions.  If more area types get added, appropriate conversions should also be added.

The following operations return a force type:
\begin{lstlisting}
function "*"(Left : mass_kg; Right : accel_m_s2) return force_n;
function "*"(Left : accel_m_s2; Right : mass_kg) return force_n;
\end{lstlisting}

\subsection{Temperature Types}
Note that range limits are not given for temperatures since the difference between two temperatures may exceed the range.  The following temperature types are defined:
\begin{lstlisting}
type temp_k is new Float;  -- temperature in kelvin
type temp_c is new Float;  -- temperature in celsius
type temp_f is new Float;  -- temperature in Fahrenheit
\end{lstlisting}

The following conversions are defined to the temperature types:
\begin{lstlisting}
function to_Farenheit(temp : temp_c) return temp_f;
function to_Kelvin(temp : temp_c) return temp_k;
function to_Celsius(temp : temp_f) return temp_c;
function to_Celsius(temp : temp_k) return temp_c;
\end{lstlisting}

No operations are currently defined that return a temperature type.

\subsection{Pressure Types}
The following pressure types are defined:
\begin{lstlisting}
type press_p is new Float;     -- pressure in pascals
type press_mb is new Float;    -- pressure in millibars
type press_atm is new Float;   -- pressure in atmospheres
type press_inHg is new Float;  -- pressure in inches of mercury
\end{lstlisting}

The following conversions are defined for pressure types:
\begin{lstlisting}
function to_milliBar(pressure : press_p) return press_mb;
function to_Atmosphere(pressure : press_p) return press_atml;
function to_inHg(pressure : press_p) return press_inHgl;
function to_Pascal(pressure : press_mb) return press_p;
function to_Pascal(pressure : press_atm) return press_p
function to_Pascal(pressure : press_inHg) return press_p;
\end{lstlisting}

No operations are currently defined that return a temperature type.

\subsection{Velocity Types}
The following velocity types are defined:
\begin{lstlisting}
type vel_m_s is new Float;    -- velocity in meters/second
type vel_mph is new Float;    -- velocity in miles per hour
type vel_km_h is new Float;   -- velocity in kilometers/hour
type vel_knots is new Float;  -- velocity in knots
\end{lstlisting}

The following conversions are defined for velocity types:
\begin{lstlisting}
function to_mph(vel : vel_m_s) return vel_mph;
function to_km_h(vel : vel_m_s) return vel_km_h;
function to_knots(vel : vel_m_s) return vel_knots;
function to_m_s(vel : vel_knots) return vel_m_s;
function to_m_s(vel : vel_km_h) return vel_m_s;
function to_m_s(vel : vel_mph) return vel_m_s;
\end{lstlisting}

The following operations return a velocity type:
\begin{lstlisting}
function "/"(Left : len_m; Right : Duration) return vel_m_s
     with Global => null, pre => (Right /= 0.0);
function "*"(Left : accel_m_s2; Right : Duration) return vel_m_s;
function "*"(Left : Duration; Right : accel_m_s2) return vel_m_s;
\end{lstlisting}

\subsection{Acceleration Types}
The following acceleration types are defined:
\begin{lstlisting}
type accel_m_s2 is new Float;  -- acceleration in meters per second squared
type accel_g is new Float;     -- acceleration in units of Earth gravity
\end{lstlisting}

The following conversions are defined for acceleration types:
\begin{lstlisting}
function to_m_s2(accel : accel_g) return accel_m_s2;
function to_g(accel : accel_m_s2) return accel_g;
\end{lstlisting}

The following operations return an acceleration type:
\begin{lstlisting}
function "/"(Left : force_n; Right : mass_kg) return accel_m_s2
     with Global => null, pre => (Right /= 0.0);
function "/"(Left : vel_m_s; Right : Duration) return accel_m_s2
     with Global => null, pre => (Right /= 0.0);
\end{lstlisting}

\subsection{Angular Types}
The following angle types are defined:
\begin{lstlisting}
type ang_r is new Float;  -- angle in radians
type ang_d is new Float;  -- angle in degrees
\end{lstlisting}

The following conversions are defined for angle types:
\begin{lstlisting}
function to_degrees(ang : ang_r) return ang_d;
function to_radians(ang : ang_d) return ang_r;
\end{lstlisting}

The following operations return an angle type:
\begin{lstlisting}
function "*"(Left : rot_d_s; Right : Duration) return ang_d;
function "*"(Left : Duration; Right : rot_d_s) return ang_d;
\end{lstlisting}

\subsection{Rotation Rate Types}
The following rotation rate types are defined:
\begin{lstlisting}
type rot_r_s is new Float;  -- rotation in radians per second
type rot_d_s is new Float;  -- rotation in degrees per second
\end{lstlisting}

The following conversions are defined for rotation rate types:
\begin{lstlisting}
function to_r_s(rot : rot_d_s) return rot_r_s;
function to_d_s(rot : rot_r_s) return rot_d_s;
\end{lstlisting}

No operations currently return a rotation rate type:

\subsection{Magnetic Types}
The following magnetic type is defined:
\begin{lstlisting}
type mag_g is new Float;  -- magnetic field in gauss
\end{lstlisting}

Since only one magnetic type is defined, there are no conversions.  If more area types get added, appropriate conversions should also be added.

No operations currently return a magnetic type.

\subsection{Electromotive Force Types}
The following electromotive force type is defined:
\begin{lstlisting}
type emf_v is new Float;  -- electromotive force in volts
\end{lstlisting}

Since only one electromotive force type is defined, there are no conversions.  If more area types get added, appropriate conversions should also be added.

The following operations return an electromotive force type:
\begin{lstlisting}
function "*"(Left : curr_a; Right : res_o) return emf_v;
function "*"(Left : res_o; Right : curr_a) return emf_v;
\end{lstlisting}

\subsection{Electric Current Types}
The following electric current type is defined:
\begin{lstlisting}
type curr_a is new Float;  -- electrical current in amps
\end{lstlisting}

Since only one electric current type is defined, there are no conversions.  If more area types get added, appropriate conversions should also be added.

The following operation returns an electric current type:
\begin{lstlisting}
function "/"(Left : emf_v; Right : res_o) return curr_a
     with Global => null, pre => (Right /= 0.0);
\end{lstlisting}

\subsection{Electrical Resistance Types}
The following electrical resistance type is defined:
\begin{lstlisting}
type res_o is new Float;  -- electrical resistance in ohms
\end{lstlisting}

Since only one electrical resistance type is defined, there are no conversions.  If more area types get added, appropriate conversions should also be added.

The following operation returns an electrical resistance type:
\begin{lstlisting}
function "/"(Left : emf_v; Right : curr_a) return res_o
     with Global => null, pre => (Right /= 0.0);
\end{lstlisting}

\subsection{Timing Related Types}
Note that Ada has a predefined Duration type that is a fixed point type.  \datatype{time\_s} is defined as a subtype of this.  The other times (minutes and hours) are derivative types so as to maintain similar precision.  If needed, they could be changed to \datatype{Float} or something else.  The following timing related types are defined:
\begin{lstlisting}
type freq_hz is new Float;    -- frequency in Hertz
subtype time_s is Duration;   -- time in seconds
-- (use subtype because seconds is identical to duration)
type time_m is new Duration;  -- time in minutes
type time_h is new Duration;  -- time in hours
\end{lstlisting}

The following conversions are defined for timing related types:
\begin{lstlisting}
function to_hz(period : time_s) return freq_hz;
function to_minutes(period : time_s) return time_m;
function to_hours(period : time_s) return time_h;
function to_seconds(freq : freq_hz) return time_s
     with Global => null, pre => (freq /= 0.0);
function to_seconds(period : time_m) return time_s;
function to_seconds(period : time_h) return time_s;
\end{lstlisting}


\end{document}
